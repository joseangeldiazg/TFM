%---------------------------------------------------
% Nombre: capitulo9.tex  
% 
% Texto del capitulo 9
%---------------------------------------------------

\chapter{Conclusiones}
\label{conclusiones}

Este cap�tulo cierra el trabajo de varios meses en la materia y se da por finalizada la memoria de este proyecto. Como el propio t�tulo indica, se comentar�n las conclusiones obtenidas en el transcurso del trabajo desde el punto de vista de un peque�o resumen del trabajo realizado y el an�lisis de las v�as futuras de investigaci�n que este trabajo de fin de grado abre y ampl�a. 

\section{Resumen del trabajo realizado}

\section{Conclusiones finales}

\section{L�neas futuras}


Tal y como se habl� en la secci�n \ref{obtenciondatos}, ser�a interesante la extensi�n del proyecto a un enfoque en la nube, de manera que se pudiera mantener este en ejecuci�n en m�quinas virtuales de alg�n proveedor de servicios cloud, como puede ser Amazon. Esto, eliminar�a las restricciones de las m�quinas personales, ya que prescindir de la misma y dejarla ejecutando grandes franjas de tiempo es inviable  y permitir�an un an�lisis m�s detallado, por ejemplo manteniendo la b�squeda de Tweets en streaming y estos siendo almacenados directamente en otra m�quina virtual en la nube con MongoDB a la que podr�amos acceder en remoto y analizar los datos eliminando el proceso de carga y sincronizaci�n y restringiendo este la configuraci�n inicial de la arquitectura. 

Por otro lado, el dataset generado es muy rico en informaci�n y se presta a la ampliaci�n del problema y su enriquecimiento, como por ejemplo, en lugar de eliminar los tuits cuyo valor de \textit{is\_rt} sea igual a \textit{true} estos pueden ser considerados como pesos y darle m�s valor, ya que en pr�cticamente la totalidad de los casos un RT implicar� que se est� de acuerdo con una opini�n. Teniendo tambi�n los id de tuit y usuario, podr�a incluso montarse una red de grafos en el que se pudiera ver en el espacio  la red de usuarios formada en torno a una opini�n o tendencia. 

Una �ltima ampliaci�n del trabajo que se ha abierto recientemente y que ofrece grandes posibilidades es la comparaci�n de resultados obtenidos en este trabajo, con el mismo proceso si realiz�ramos la obtenci�n de los datos ahora mismo, donde los tuits en lugar de 140 caracteres pueden llegar a ocupar 280. Cave de esperar por tanto un aumento de palabras de opini�n y sentimientos, que al menos ser�a m�s elaboradas y podr�an aportar mucha m�s informaci�n de cara a un nuevo estudio. 

\pagebreak
\clearpage
%---------------------------------------------------