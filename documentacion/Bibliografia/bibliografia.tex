%---------------------------------------------------
% Nombre: bibliografia.tex  
% 
% Bibliografia
%---------------------------------------------------

\begin{thebibliography}{99}


\bibitem{com}
	\emph{}Moosavi, S.A. and Jalali, M. Community detection in online social networks using actions of users. 2014 \textit{Iranian Conference on Intelligent Systems, ICIS}.

\bibitem{sent}
	\emph{}K. Kwon, Y. Jeon, C. Cho, J. Seo, In-Jeong Chung, H. Park: Sentiment trend analysis in social web environments. \textit{BigComp 2017}, 261-268

\bibitem{sent2}
	\emph{}M. Pilar Salas-Z�rate, J. Medina-Moreira, K. Lagos-Ortiz, H. Luna-Aveiga, M. �ngel Rodr�guez-Garc�a, R. Valencia-Garc�a: Sentiment Analysis on Tweets about Diabetes: An Aspect-Level Approach. \textit{Comp. Math. Methods in Medicine} 2017. 
	
\bibitem{analitica}	
	\emph{}Serrano-Cobos, Jorge. Big data y anal�tica web. Estudiar las corrientes y pescar en un oc�ano de datos. \textit{El profesional de la informaci�n}, 2014, vol. 23, n. 6, pp. 561-565.

\bibitem{finance}
	\emph{}E. W. T. Ngai, Yong Hu, Y. H. Wong, Yijun Chen, and Xin Sun. 2011. The application of data mining techniques in financial fraud detection: A classification framework and an academic review of literature. Decis. Support Syst. 50, 3 (February 2011), 559-569. 

\bibitem{liu1}
	\emph{}Pritam Gundecha, Huan Liu. Mining Social Media: A Brief Introduction. Arizona State University, Tempe, Arizona.

\bibitem{liu}
	\emph{}B Liu, L Zhang . A survey of opinion mining and sentiment analysis. \textit{Mining text data}, 2012. Springer.
	
\bibitem{diff}
	\emph{}S. Noferesti, and  M. Shamsfard. Resource Construction and Evaluation for Indirect Opinion Mining of Drug Reviews. \textit{PLOS ONE}, 2015. 
	

\bibitem{diff2}
	\emph{}Cambria E, Speer R, Havasi C, Hussain A. SenticNet: A publicly available semantic resource for opinion mining. \textit{ AAAI CSK}. 2010, 14-8.

	
\bibitem{mkt}
	\emph{}Baier D., Daniel I.  Image Clustering for Marketing Purposes. In: Gaul W., Geyer-Schulz A., Schmidt-Thieme L., Kunze J. Challenges at the Interface of Data Analysis, Computer Science, and Optimization. \textit{Studies in Classification, Data Analysis, and Knowledge Organization}. Springer, Berlin, Heidelberg. 2012. 

\bibitem{agrawal}
	\emph{}Rakesh Agrawal, Tomasz Imieliski, and Arun Swami. Mining association rules between sets of items in large databases. \textit{SIGMOD} Rec. 22, 1993, 207-216. 

\bibitem{reglas1}
	\emph{}P. Mandave, M. Mane, S. Patil. Data mining using Association rule based on APRIORI algorithm and improved approach with illustration. \textit{ International Journal of Latest Trends in Engineering and Technology (IJLTET)}, Vol. 3 Issue2 November 2013.

\bibitem{reglas2}
	\emph{}Yong Yin, Ikou Kaku, Jiafu Tang, JianMing Zhu. Data Mining. Chapter 2, Association Rules Mining in Inventory Database (pp 9-23). Springer, 2011. 


\bibitem{sentimientos}
	\emph{}R. Dehkharghani, H. Mercan, A. Javeed, Y. Saygin: Sentimental causal rule discovery from Twitter. \textit{Expert Syst. Appl.} 41(10): 4950-4958 (2014).

\bibitem{bdm1}
	\emph{}S. M. Weiss and N. Indurkhya. \textit{Predictive data mining: a practical guide.} Morgan Kaufmann Publishers Inc., San Francisco, CA, USA, 1998.

 \bibitem{bdm2}
	\emph{}A. Petland. Reinventing society in the wake of big data. Edge.org, \url{http://www.edge.org/conversation/reinventing-society- in-the-wake-of-big-data}, 2012.
	
\bibitem{bdm3}
	\emph{}D. Laney. 3-D Data Management: Controlling Data Volume, Velocity and Variety. \textit{META Group Research Note, February 6}, 2001.

\bibitem{mining1}	
	\emph{}Han, J.W. and Kamber, M. (2001) Data Mining: Concepts and Techniques. Morgan Kaufmann Pulishers, Inc., San Francisco.

\bibitem{mining2}
	\emph{}Tan, P.N., Steinbach, M. and Kumar, V. (2006) Introduction to Data Mining. Pearson Education, Inc., London, 30-336.	
	
\bibitem{mining3}
	\emph{}W. Seo, J. Yoon, H. Park, B. Coh, J. Lee, O. Kwon. Product opportunity identification based on internal capabilities using text mining and association rule mining. \textit{Technological Forecasting \& Social Change} 105 (2016) 94-104.	
	
\bibitem{mining4}
	\emph{}M. Kaura, S. Kanga. Market Basket Analysis: Identify the changing trends of market data using association rule mining. International Conference on Computational Modeling and Security (CMS 2016). \textit{Procedia Computer Science} 85 (2016) 78 - 85. 
		
\bibitem{bigdata1}	
	\emph{}K. Jayabal, Dr. P. Marikkannu. An Efficient Big Data processing for frequent itemset mining based on MapReduce Framework.  International Journal of Novel Research in Computer Science and Software Engineering Vol. 3, Issue 1, pp: (130-134).

\bibitem{bigdata5}	
	\emph{}Lin, Ming-Yen and Lee, Pei-Yu and Hsueh, Sue-Chen. Apriori-based Frequent Itemset Mining Algorithms on MapReduce. \textit{ICUIMC} , 2012. pp(76:1-76:8).

\bibitem{bigdata2}
	\emph{}X. Zhou and Y. Huang. An improved parallel association rules algorithm based on MapReduce framework for big data. 11th International Conference on Fuzzy Systems and Knowledge Discovery (FSKD), Xiamen, 2014, pp. 284-288.

\bibitem{bigdata3}
	\emph{}Y. Chen, F. Li, J. Fan. Mining association rules in big data with NGEP. \textit{Cluster Computin}, 2015, 18:2, 577-585.
	
\bibitem{bigdata4}
	\emph{}Dr. R Nedunchezhian and K Geethanandhini. Association Rule Mining on Big Data. International Journal of Engineering Research \& Technology (IJERT). Volume. 5 - Issue. 05. (2015).
	
\bibitem{redes1}
	\emph{}M Adedoyin-Olowe, M Medhat Gaber, Frederic T. Stahl: A Survey of Data Mining Techniques for Social Media Analysis. \textit{JDMDH} 2014.


\bibitem{redes2}
	\emph{}YZhou, N Sani, Chia-Kuei Lee, J Luo: Understanding Illicit Drug Use Behaviors by Mining Social Media. \textit{CoRR} abs/1604.07096 (2016).
	

	
\bibitem{redes3}
	\emph{}L Cagliero and A Fiori. Analyzing Twitter User Behaviors and Topic Trends by Exploiting Dynamic Rules. Behavior Computing: Modeling, Analysis, Mining and Decision. Springer, 2012  pp. 267-287.
	
	\bibitem{redes4}
	\emph{}L. Maria Aiello, G Petkos, Carlos J. Mart�n, D Corney, S Papadopoulos, R Skraba, A G�ker, I Kompatsiaris, A Jaimes: Sensing Trending Topics in Twitter. \textit{IEEE Trans}. Multimedia 15(6): 1268-1282 (2013).
	
\bibitem{estadoarte4}
	\emph{}X Yu, S Miao, H Liu, Jenq-Neng Hwang, W Wan, J Lu: Association Rule Mining of Personal Hobbies in Social Networks. \textit{Int. J. Web Service Res.} 14: 13-28 (2017).
		
\bibitem{estadoarte5}
	\emph{}F Erlandsson, P Br�dka, A Borg, H Johnson: Finding Influential Users in Social Media Using Association Rule Learning. Entropy 18: 164 (2016).

\bibitem{opiniontwitter}
	\emph{}A Pak, P Paroubek. Twitter as a Corpus for Sentiment Analysis and Opinion Mining. \textit{Lrec}. 2010. 

\bibitem{opmin1}
	\emph{}Ana M. Popescu and O Etzioni. Extracting product features and opinions from reviews. \textit{HLT '05 Proceedings of the conference on Human Language Technology and Empirical Methods in Natural Language Processing} Pages 339-346. 2005. 


\bibitem{opmin2}
	\emph{} Hai Z., Chang K., Kim J. (2011) Implicit Feature Identification via Co-occurrence Association Rule Mining. In: Gelbukh A.F. (eds) \textit{Computational Linguistics and Intelligent Text Processing. CICLing 2011.} Lecture Notes in Computer Science, vol 6608. Springer, Berlin, Heidelberg


\bibitem{opmin3}
	\emph{}Yuan M., Ouyang Y., Xiong Z., Sheng H. (2013) Sentiment Classification of Web Review Using Association Rules. In: Ozok A.A., Zaphiris P. (eds) \textit{Online Communities and Social Computing. OCSC 2013.�} Lecture Notes in Computer Science, vol 8029. Springer, Berlin, Heidelberg

\bibitem{estadoarte1}
	\emph{}Z Farzanyar, N Cercone: Efficient mining of frequent itemsets in social network data based on MapReduce framework. \textit{ASONAM} 2013: 1183-1188.


\bibitem{estadoarte2}
	\emph{}S. Gole and B. Tidke, Frequent itemset mining for Big Data in social media using ClustBigFIM algorithm. International Conference on Pervasive Computing (ICPC), Pune, 2015, pp. 1-6.
	
	
\bibitem{bigfim}
	\emph{}S. Moens, E. Aksehirli, B. Goethals: Frequent Itemset Mining for Big Data. BigData Conference 2013: 111-118.
	
\bibitem{estadoarte3}
	\emph{}J Yang and B Yecies. Open AccessMining Chinese social media UGC: a bigdata framework for analyzing Douban movie reviews, 2016, \textit{Journal of Big Data}, vol 1. 

\bibitem{estadoarte4}
	\emph{} Abascal-Mena R., L�pez-Ornelas E., Zepeda-Hern�ndez J.S. (2013) User Generated Content: An Analysis of User Behavior by Mining Political Tweets. In: Ozok A.A., Zaphiris P. (eds) Online Communities and Social Computing. OCSC 2013. Lecture Notes in Computer Science, vol 8029. Springer, Berlin, Heidelberg

 \bibitem{sentiwordnet}
	\emph{}Esuli, A., Sebastiani, F.: SENTIWORDNET: A Publicly Available Lexical Resource for Opinion Mining. \textit{Proceedings of the 5th Conference on Language Resources and Evalua- tion}, LREC 2006, Genova, Italy, pp. 417?422 (2006)

\bibitem{grafos}
	\emph{} D. Ediger, K. Jiang, J. Riedy, D. A. Bader and C. Corley, "Massive Social Network Analysis: Mining Twitter for Social Good," 2010 39th International Conference on Parallel Processing, San Diego, CA, 2010, pp. 583-593.
	
\bibitem{mongo}
	\emph{} Web del proyecto MongoDB. \url{https://www.mongodb.com}

\bibitem{tweepy}
	\emph{} Web del proyecto Tweepy. \url{http://www.tweepy.org}
	
\bibitem{scrapy}
	\emph{} Web de Scrapy. \url{https://scrapy.org}	
	
\bibitem{scrapinghub}
	\emph{} Web de Scrapinghub. \url{https://scrapinghub.com}	
	
\bibitem{sparkr}
	\emph{} Web de SparkR \url{https://spark.apache.org/docs/latest/sparkr.html}
		
\bibitem{tm}
	\emph{} I. Feinerer ,K. Hornik. (2017) Text Mining Package (Versi�n 0.7-3) [Software] Recuperado de \url{https://cran.r-project.org/}
		
\bibitem{nlp}
	\emph{} Manning, Christopher D., Mihai Surdeanu, John Bauer, Jenny Finkel, Steven J. Bethard, and David McClosky. 2014. The Stanford CoreNLP Natural Language Processing Toolkit In \textit{Proceedings of the 52nd Annual Meeting of the Association for Computational Linguistics: System Demonstrations}, pp. 55-60.		
	
\bibitem{ner}
	\emph{}Jenny Rose Finkel, Trond Grenager, and Christopher Manning. 2005. Incorporating Non-local Information into Information Extraction Systems by Gibbs Sampling. Proceedings of the 43nd Annual Meeting of the Association for Computational Linguistics (ACL 2005), pp. 363-370.
		
\end{thebibliography}
%---------------------------------------------------