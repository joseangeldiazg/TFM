%---------------------------------------------------
% Nombre: capitulo2.tex  
% 
% Texto del cap�tulo 2
%---------------------------------------------------

\chapter{Planificaci�n del proyecto}


\section{Modelo de desarrollo}


\section{Gesti�n de recursos}


\subsection{Personal}

\subsection{Hardware}

\subsection{Software}


\section{Planificaci�n temporal}

La parte m�s importante de esta secci�n radica en la planificaci�n temporal seguida en los meses de trabajo que el proyecto ha ocupado, siendo este elaborado continuamente etapa a etapa. 

\begin{enumerate}
\item Obtenci�n de informaci�n y estudio del tema: La primera parte del proyecto consisti� en la obtenci�n de informaci�n acerca de las reglas de asociaci�n y la aplicaci�n de estas en el �mbito de la miner�a de redes sociales y m�s concretamente en Twitter. En este primer proceso de recopilaci�n de informaci�n tambi�n se estudiaron temas m�s gen�ricos dentro del Big Data y la miner�a de datos con el fin de tener una visi�n global de las herramientas y t�cnicas a estudiar y usar en el problema. Esta etapa aunque ha sido continua, tuvo especial importancia desde mediados de noviembre de 2016 a finales de diciembre de ese mismo a�o. 

\item Estudio del estado del arte: Tras obtener buena cantidad de informaci�n y comprender el problema a resolver se dio comienzo a desarrollar un estudio exhaustivo del estado del arte de la materia as� como a comenzar a desarrollar los primeros cap�tulos de la memoria en cuesti�n. Esta etapa tuvo lugar desde finales de diciembre de 2016 hasta finalizar el proyecto debido a que se ha realizado un estudio continuo de los nuevos estudios que iban apareciendo sobre la tem�tica. 
 
 \item Selecci�n de herramientas: Una vez fijado Twitter como medio objetivo, se llev� a cabo una investigaci�n sobre las herramientas m�s oportunas para la obtenci�n de los tuits de la red social. Esta etapa tomo lugar entre final de junio y principio de julio de 2017. 
 
\item Obtenci�n del dataset: Para poder comenzar a hacer pruebas y desarrollar el sistema basado en reglas, una vez elegida la herramienta, (Scrapy), se comenz� a obtener datos de la red social durante unos d�as ininterrumpidamente para tener un conjunto de entrenamiento suficiente. Esta tarea tomo lugar a mediados de julio de 2017. 

\end{enumerate}

\section{Costes}

\pagebreak

\clearpage
%---------------------------------------------------