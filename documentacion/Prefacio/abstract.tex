%---------------------------------------------------
% Nombre: presentacion.tex  
% 
% Texto de presentaci�n del proyecto
%---------------------------------------------------

\mbox{}
\begin{center}
{\large\bfseries Trend Analysis with Big Data}\\
\end{center}

\noindent{\textbf{Keywords}: Association rules, Big Data, text mining, Twitter}\\


{\Large \textbf{Abstract:}}

It can be said that in the last decade we have generated more data than ever before in history. This is not a farfetched assertion if we note that, nowadays, practically everything generates data from which we can most likely extract knowledge that will eventually turn into a competitive advantage. That is why the techniques of data mining, are becoming more popular and being studied both entrepreneurially and academically.

In the area of data mining, we find the field of trend analysis or opinion mining, which has been widely studied in the last decade. One of the reasons for the growing interest of the subject, lies in the emergence of social networks and the exponential growth in the number of users of these. In this way we have provided a large database where we can obtain valuable information in a way that allows companies or organizations of any kind to know, for example, if a new product has received good reviews among its customers. It is in this last point where this study focuses, deepened in the study of unsupervised learning techniques and their application in the field of text and opinion mining.

Through these techniques, and the result of the combination of several of them, we will try to obtain a model that feeds data obtained from the social network Twitter, processes them applying text mining techniques, polarizes according to classification based on feelings and finally offers output easily interpretable as opinion trends in the social network.

\clearpage
%---------------------------------------------------