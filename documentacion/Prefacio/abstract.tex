%---------------------------------------------------
% Nombre: presentacion.tex  
% 
% Texto de presentaci�n del proyecto
%---------------------------------------------------

\mbox{}
\begin{center}
{\large\bfseries Trend Analysis with Big Data}\\
\end{center}

\noindent{\textbf{Keywords}: Association rules, Big Data, text mining, Twitter}\\


{\Large \textbf{Abstract:}}

It can be said that in the last decade we have generated more data than ever before in history. This is not a farfetched assertion if we note that even in our daily routines, such as shopping or sharing photos with friends through a social network, we are generating a lot of data; and nowadays, practically everything generates a series of data from which we can most likely extract knowledge that will eventually turn into a competitive advantage. That is why the techniques of data mining and machine learning, although they are not new, are becoming more popular and being studied both entrepreneurially and academically.

On the other hand, if we add to the power of these traditional techniques of data mining the novelty of Big Data paradigms and that large open database provided by social networks, we can obtain models that help to understand patterns of behavior and even predict future acts with very good precision.

In this document we can find the result of several months of study in data mining, in the course of which we have dig in the application of unsupervised learning techniques in the problem of text mining to later develop a model based on association rules that allows us to obtain relevant information about people that could have a special relevance in the social network Twitter. Lastly, the rules have been categorized from a subjective approach based on sentiment analysis, giving another view beyond what the expert might have about the rules and condensing the information obtained from 1.7M of tweets in a set of easily understandable association rules.
\clearpage
%---------------------------------------------------