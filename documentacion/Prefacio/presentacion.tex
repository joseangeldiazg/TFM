%---------------------------------------------------
% Nombre: presentacion.tex  
% 
% Texto de presentaci�n del proyecto
%---------------------------------------------------

\mbox{}
\begin{center}
{\large\bfseries An�lisis de tendencias con Big Data}\\
\end{center}

\noindent{\textbf{Palabras clave}: Reglas de Asociaci�n, Big Data, miner�a de textos, Twitter}\\

{\Large \textbf{Resumen:}}

Podemos afirmar que en la �ltima d�cada de nuestra historia se ha generado mayor cantidad de datos que en la totalidad de la historia restante. Esta no es una afirmaci�n descabellada si notamos que hasta en nuestros h�bitos diarios como hacer la compra o compartir fotos con amigos a trav�s de alguna red social estamos generando una gran cantidad de datos y es que hoy en d�a pr�cticamente todo genera una serie de datos de los cuales muy probablemente podamos extraer conocimiento que se acabar� traduciendo en una ventaja competitiva. Es por ello que las t�cnicas de miner�a de datos y aprendizaje autom�tico, aunque no son algo nuevo, est�n en alza y cada vez son m�s estudiadas tanto empresarial como acad�micamente. 

Si por otro lado a la potencia de estas t�cnicas tradicionales de miner�a de datos le sumamos la reciente novedad de los paradigmas del Big Data y esa gran base de datos abierta que proporcionan las redes sociales, pueden obtenerse modelos que ayuden a comprender patrones de comportamiento e incluso siendo bastante optimistas predecir con muy buena precisi�n actos futuros. 

En este documento podemos encontrar el resultado de varios meses de estudio en la materia, en el transcurso de los cuales se ha profundizado en la aplicaci�n de t�cnicas de aprendizaje no supervisado en el problema de la miner�a de textos para posteriormente desarrollar un modelo basado en reglas de asociaci�n que nos permite la obtenci�n de informaci�n relevante sobre personas que pudieran tener una especial relevancia en la red social Twitter. En �ltimo lugar, se han categorizado las reglas desde un enfoque subjetivo basado en an�lisis de sentimientos otorgando otra visi�n m�s all� de la que el experto pudiera tener sobre las reglas y condensando la informaci�n obtenida de 1.7M de tuits en un set de reglas f�cilmente comprensible. 

\clearpage
%---------------------------------------------------