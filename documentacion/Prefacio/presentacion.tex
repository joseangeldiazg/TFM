%---------------------------------------------------
% Nombre: presentacion.tex  
% 
% Texto de presentaci�n del proyecto
%---------------------------------------------------

\mbox{}
\begin{center}
{\large\bfseries An�lisis de tendencias con Big Data}\\
\end{center}

\noindent{\textbf{Palabras clave}: Reglas de Asociaci�n, Big Data, miner�a de textos, Twitter}\\

{\Large \textbf{Resumen:}}

Podemos afirmar que en la �ltima d�cada de nuestra historia se ha generado mayor cantidad de datos que en la totalidad de la historia restante. Esta no es una afirmaci�n descabellada si denotamos que hoy en d�a pr�cticamente cada actividad que realizamos genera una serie de datos de los cuales muy probablemente podamos extraer conocimiento que se acabar� traduciendo en una ventaja competitiva. Es por ello que las t�cnicas de miner�a de datos, est�n en alza y cada vez son m�s estudiadas tanto empresarial como acad�micamente. 

Dentro de la miner�a de datos encontramos el campo del an�lisis de tendencias o miner�a de opiniones, que ha sido ampliamente estudiada en la �ltima d�cada. Uno de los motivos del creciente inter�s de la materia, reside en la aparici�n de las redes sociales y el crecimiento exponencial del n�mero de usuarios de las mismas. De este modo se ha proporcionado una gran base de datos de donde poder obtener informaci�n de valor de manera que permita a empresas u organizaciones de cualquier tipo conocer, por ejemplo, si un nuevo producto ha recabado buenas cr�ticas entre su clientela. Es en este �ltimo punto donde se centra este trabajo, en el que se ha profundizado en el estudio de las t�cnicas de aprendizaje no supervisado y su aplicaci�n en el �mbito de la miner�a de textos y opiniones. 

Por medio de estas t�cnicas, y fruto de la combinaci�n de varias de ellas, se tratar� de obtener un modelo que se nutre de datos obtenidos de la red social Twitter, los procesa aplicando t�cnicas de miner�a de textos, polariza en funci�n de clasificaci�n basada en sentimientos y por �ltimo ofrece como salida resultados f�cilmente interpretables como tendencias de opini�n en la red social. 

\clearpage
%---------------------------------------------------